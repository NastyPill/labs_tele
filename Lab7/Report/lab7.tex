\documentclass[10pt,a4paper,oneside]{article}
\usepackage{cmap}
\usepackage[T2A]{fontenc}
\usepackage{float}
\usepackage{listings}
\usepackage{csquotes}
\usepackage[utf8]{inputenc}
\usepackage{amsmath}
\usepackage{amsfonts}
\usepackage{amssymb}
\usepackage[english, russian]{babel}%Подключаем русский язык.
\usepackage{graphicx}
\usepackage{geometry} % Меняем поля страницы.
\geometry{left=3cm} %Левое поле.
\geometry{right=2cm} %Правое поле.
\geometry{top=3cm} %Верхнее поле.
\geometry{bottom=2cm} %Нижнее поле.


%Начало документа
\begin{document}

%Создаём титульник.
\begin{titlepage}
\newpage
	%Название ВУЗа и институт.
	\begin{center}
		\Large Санкт-Петербургский Государственный Политехнический Университет\\
		Институт Компьютерных Наук и Технологий\\
	\end{center}
	%Кафедра.
	\begin{center}
		\large\textbf {Высшая школа интеллектуальных систем и суперкомпьютерных технологий}
	\end{center}
	
	%Пропуск места. 
	\vspace{5em}
	%!!!!!!!!!!!!!!!!!!!!!!!!!!!!!!!!!Название работы.
	\begin{center}
		\large{Отчёт по лабораторной работе №7 \\ на тему \\
		\textbf{Дискретное преобразование фурье} }
	\end{center}
	
	%Делаем пропуск и пишем студента и преподавателя.
	\vspace{25em}
	\begin{flushright}
		\textbf{Работу выполнил\\}Студент группы 3530901/80203 \\ Танашкин В.А.\\
		\textbf{Преподаватель\\}Богач Н.В. 
	\end{flushright}
	
	\vspace{\fill}%В самом низу
	\begin{center}
	Санкт-Петербург, 2021 год	
	\end{center}
\end{titlepage} %Закончили титульный лист.

\section{Настройка проекта}
Перед тем как выполнять задания необходимо настроить проект и сделать все необходимые импорты:

\begin{figure}[H]
        \centering
        \includegraphics[width=0.75\textwidth]{pics/0.png}
        \caption{2}
        \label{fig:first}
\end{figure}

\section{Ход работы}

Возьмем небольшой сигнал и вычислим его DFT:

\begin{figure}[H]
        \centering
        \includegraphics[width=0.75\textwidth]{pics/1.png}
        \caption{2}
        \label{fig:first}
\end{figure}

Реализация DFT из книги:

\begin{figure}[H]
        \centering
        \includegraphics[width=0.75\textwidth]{pics/2.png}
        \caption{2}
        \label{fig:first}
\end{figure}

Проведем тест, для того, чтобы убедиться что результаты совпадают с использованием np.fft.fft для вычисления DFT

\begin{figure}[H]
        \centering
        \includegraphics[width=0.75\textwidth]{pics/3.png}
        \caption{2}
        \label{fig:first}
\end{figure}

Для начала проектирования рекурсивного DFT я реализую метод, который разбивает входной массив и использует np.fft.fft для вычисления DFT половин.

\begin{figure}[H]
        \centering
        \includegraphics[width=0.75\textwidth]{pics/4.png}
        \caption{2}
        \label{fig:first}
\end{figure}

Получаем такие результаты:

\begin{figure}[H]
        \centering
        \includegraphics[width=0.75\textwidth]{pics/5.png}
        \caption{2}
        \label{fig:first}
\end{figure}

Наконец, мы можем заменить np.fft.fft рекурсивными вызовами и добавить базовый вариант:

\begin{figure}[H]
        \centering
        \includegraphics[width=0.75\textwidth]{pics/6.png}
        \caption{2}
        \label{fig:first}
\end{figure}

Эта реализация DFT требует времени, пропорционального «log». Это также занимает пространство, пропорциональное "log", и тратит время на создание и копирование массивов. Его можно улучшить, чтобы он работал «на месте»; в этом случае он не требует дополнительного места и тратит меньше времени на накладные расходы.

\end{document}
